\documentclass[11pt,a4paper]{article}

\usepackage[margin=1in]{geometry}
\usepackage{booktabs}
\usepackage{hyperref}
\usepackage{xcolor}
\usepackage{enumitem}
\usepackage{fancyhdr}
\usepackage{titlesec}
\usepackage{longtable}
\usepackage{amsmath}
\usepackage{graphicx}
\usepackage{listings}
\usepackage{caption}

\definecolor{improved}{RGB}{34,139,34}
\definecolor{regression}{RGB}{178,34,34}
\definecolor{neutral}{RGB}{70,70,70}
\definecolor{codebg}{RGB}{245,245,245}

\lstset{
  basicstyle=\ttfamily\small,
  backgroundcolor=\color{codebg},
  frame=single,
  breaklines=true,
  columns=flexible,
}

\pagestyle{fancy}
\fancyhf{}
\rhead{DA3 Costmap Ablation}
\lhead{NCHSB Project}
\rfoot{Page \thepage}

\title{%
  \textbf{DA3 Depth-Based Costmap Ablation Study} \\[0.5em]
  \large Evaluating DA3-Driven Local Costmaps and Dynamic Inflation for Nav2 \\[0.3em]
  \normalsize NYU MS Project --- Nishant Prabhu
}
\author{}
\date{March 2026}

\begin{document}
\maketitle
\tableofcontents
\newpage

% ============================================================================
\section{Study Overview}
% ============================================================================

This document records the results of a depth-based costmap ablation study
evaluating whether DA3-Small monocular depth predictions can improve Nav2
local costmaps for autonomous corridor navigation.

\subsection{Motivation}

The current Nav2 stack uses only a 2D LiDAR (\texttt{/scan}) for the
obstacle layer and a fixed inflation radius of 0.09\,m. The 2D LiDAR
scans at a single height and cannot detect obstacles at other heights
(low obstacles, overhanging objects). DA3-Small produces dense depth maps
from a single RGB camera at 25--40\,FPS on Jetson, covering the full
image with depth at all heights within the camera FOV ($\sim$81\textdegree{}).

The Orbbec Femto Bolt ToF sensor has 77\% dead pixels in the deployment
corridor (reflective floors, specular surfaces), confirming that monocular
depth estimation is necessary for dense depth coverage.

\subsection{Experiment Matrix}

Seven configurations are evaluated:

\begin{table}[h]
\centering
\caption{Ablation experiment matrix}
\label{tab:experiments}
\begin{tabular}{llll}
\toprule
\textbf{Experiment} & \textbf{Obstacle Source} & \textbf{Inflation} & \textbf{Tests} \\
\midrule
Baseline & Sensor depth only     & Fixed 0.09\,m            & Current Nav2 config \\
A1       & Sensor depth only     & Dynamic (corridor width)  & Dynamic inflation alone \\
A2       & Sensor + DA3 depth    & Fixed 0.09\,m            & Added depth benefit \\
A3       & Sensor + DA3 depth    & Dynamic (corridor width)  & Best of both \\
A4       & DA3 depth only        & Dynamic (corridor width)  & LiDAR replacement \\
A5       & Sensor + DA3 depth    & Class-aware (YOLO+SAM2)  & Per-class safety \\
A6       & DA3 depth only        & Class-aware (YOLO+SAM2)  & Full vision-only stack \\
\bottomrule
\end{tabular}
\end{table}

\subsection{Metrics}

\begin{itemize}[nosep]
  \item \textbf{Costmap IoU:} agreement between experiment and Baseline occupancy
  \item \textbf{Obstacle detection rate:} \% of Baseline lethal cells also detected
  \item \textbf{False positive rate:} \% of Baseline free cells incorrectly marked
  \item \textbf{Clearance margin:} min distance from robot centre to nearest lethal cell
  \item \textbf{Inflation radius:} per-frame dynamic radius (for A1, A3, A4)
  \item \textbf{Computation time:} per-frame pipeline latency
\end{itemize}

% ============================================================================
\section{Phase 1: Offline Corridor Ablation}
\label{sec:phase1-corridor}
% ============================================================================

\subsection{Dataset}

459 frames extracted from Orbbec Femto Bolt rosbag (81\,s corridor run).
Sensor depth aligned to colour camera via 3D reprojection
(\texttt{extract\_corridor\_bag.py}). DA3-Small depth pre-computed on HPC.

Camera: $f_x = 747.8$, $f_y = 747.5$\,px. Camera height: $\sim$0.25\,m.

\subsection{Run 1 Results (Pre-Fix)}

Initial run with \texttt{corridor\_width} strategy using
$r = \text{clip}(w \times 0.05,\, 0.05,\, 0.20)$. Two issues identified
and fixed (see Section~\ref{sec:fixes}).

\begin{table}[h]
\centering
\caption{Phase 1 corridor ablation --- Run 1 (459 frames, pre-fix)}
\label{tab:phase1-run1}
\begin{tabular}{lrrrrrr}
\toprule
\textbf{Exp.} & \textbf{IoU} & \textbf{Det. Rate} & \textbf{FPR} & \textbf{Clearance} & \textbf{Radius} & \textbf{Time} \\
\midrule
Baseline & 1.000 & 1.000 & 0.000 & 0.299\,m & 0.090\,m & 17.3\,ms \\
A1       & 1.000 & 1.000 & 0.000 & 0.299\,m & 0.050\,m & 56.3\,ms \\
A2       & 0.295 & 1.000 & 0.050 & 0.220\,m & 0.090\,m & 87.3\,ms \\
A3       & 0.295 & 1.000 & 0.050 & 0.220\,m & 0.050\,m & 126.1\,ms \\
A4       & 0.226 & 0.767 & 0.050 & 0.224\,m & 0.055\,m & 121.7\,ms \\
\bottomrule
\end{tabular}
\end{table}

\textbf{Key findings (Run 1):}
\begin{itemize}[nosep]
  \item A2/A3 detection rate = 100\%: adding DA3 depth never loses sensor obstacles
  \item A4 detection rate = 76.7\%: DA3 alone misses 23\% of obstacles (limited FOV)
  \item A2/A3 clearance 0.220\,m vs Baseline 0.299\,m: DA3 detects obstacles the
    sensor misses (77\% dead ToF pixels), tightening clearance (safer)
  \item Corridor width estimates: mean 0.55\,m, all dynamic radii clamped to 0.05\,m floor
\end{itemize}

\subsection{Fixes Applied}
\label{sec:fixes}

\subsubsection{Fix 1: Dynamic Inflation Radius Scaling}

The original formula $r = \text{clip}(w \times 0.05,\, 0.05,\, 0.20)$ produced
$r = 0.05$\,m for all corridor frames (width 0.50--0.97\,m). The strategy was
inverted: narrow corridors need \emph{larger} inflation for safety, not smaller.

New mapping (inverted logic):
\begin{itemize}[nosep]
  \item $w \leq 0.5$\,m $\to$ $r = 0.18$\,m (tight corridor, max safety)
  \item $w = 1.0$\,m $\to$ $r = 0.12$\,m (moderate)
  \item $w = 2.0$\,m $\to$ $r = 0.06$\,m (wide passage)
  \item $w \geq 3.0$\,m $\to$ $r = 0.05$\,m (open space)
\end{itemize}

\subsubsection{Fix 2: Cost-Threshold-Aware Metrics}

Metrics previously compared only lethal cells (cost = 254). Inflation
creates a cost gradient (1--253) around obstacles, but this was invisible
to IoU and clearance metrics. A1 vs Baseline showed identical IoU despite
different inflation radii (0.05\,m vs 0.09\,m).

Fix: metrics now use a cost threshold of 128. Cells with cost $\geq 128$
are ``occupied'' for IoU, FPR, and clearance. A new \texttt{lethal\_iou}
metric tracks raw obstacle agreement separately.

\subsubsection{Fix 3: Class-Aware Inflation Strategy}

The \texttt{class\_aware} strategy now defines per-class radii for use
with YOLO + SAM2 detection data:

\begin{table}[h]
\centering
\caption{Per-class inflation radii (class\_aware strategy)}
\label{tab:class-radii}
\begin{tabular}{lr}
\toprule
\textbf{Class} & \textbf{Radius (m)} \\
\midrule
person     & 0.30 \\
glass      & 0.20 \\
furniture  & 0.15 \\
other      & 0.12 \\
wall       & 0.06 \\
floor      & 0.00 \\
\bottomrule
\end{tabular}
\end{table}

Falls back to \texttt{min\_depth} strategy when no detection data is available.

\subsection{Run 2 Results (Post-Fix)}

\begin{table}[h]
\centering
\caption{Phase 1 corridor ablation --- Run 2 (459 frames, post-fix)}
\label{tab:phase1-run2}
\begin{tabular}{lrrrrrrr}
\toprule
\textbf{Exp.} & \textbf{IoU} & \textbf{L-IoU} & \textbf{Det.Rate} & \textbf{FPR} & \textbf{Clear.} & \textbf{Radius} & \textbf{Time} \\
\midrule
Baseline & 1.000 & 1.000 & 1.000 & 0.000 & 0.249\,m & 0.090\,m & 17.7\,ms \\
A1       & 1.000 & 1.000 & 1.000 & 0.000 & 0.249\,m & 0.177\,m & 56.9\,ms \\
A2       & 0.379 & 0.295 & 1.000 & 0.053 & 0.171\,m & 0.090\,m & 89.8\,ms \\
A3       & 0.379 & 0.295 & 1.000 & 0.053 & 0.171\,m & 0.177\,m & 128.8\,ms \\
A4       & 0.279 & 0.226 & 0.767 & 0.053 & 0.175\,m & 0.165\,m & 124.8\,ms \\
\bottomrule
\end{tabular}
\end{table}

\textbf{Key findings (Run 2):}
\begin{enumerate}[nosep]
  \item \textbf{Dynamic inflation now adaptive:} corridor\_width strategy produces
    $r = 0.177$\,m (range 0.156--0.180\,m) for the narrow corridor ($w \approx 0.55$\,m),
    nearly double the fixed 0.09\,m. The inverted scaling correctly assigns larger
    inflation to tighter passages.

  \item \textbf{Sensor+DA3 fusion (A2/A3) is strictly additive:} 100\% detection rate
    confirms DA3 never removes sensor obstacles. The 5.3\% FPR represents DA3 marking
    obstacle cells where the sensor saw free space --- likely areas where the ToF sensor
    had dead pixels (77\% failure rate in corridor). This is a \emph{safety improvement},
    not noise.

  \item \textbf{DA3 alone (A4) cannot replace LiDAR:} detection rate 76.7\% means
    23.3\% of sensor-detected obstacles are missed, due to limited camera FOV
    ($\sim$81\textdegree{} vs 360\textdegree{}).

  \item \textbf{Clearance tightens with DA3:} Baseline 0.249\,m $\to$ A2/A3 0.171\,m
    ($-0.078$\,m). DA3 fills in the dead ToF pixels with valid depth, producing a
    denser and more conservative costmap. In a corridor with $\sim$0.8\,m clearance
    per side, 0.171\,m is still safe for the 0.4\,m-wide robot.

  \item \textbf{All configurations within Nav2 timing budget:} worst case 128.8\,ms
    (A3), well within the 200\,ms local costmap update period (5\,Hz publish rate).
\end{enumerate}

\subsection{Corridor Width Distribution}

Estimated corridor widths: mean $= 0.55$\,m, std $= 0.09$\,m,
range $[0.50,\, 0.97]$\,m. The narrow range reflects the uniform
corridor geometry in the 81-second rosbag. The A4 inflation radius
shows more variation (std $= 0.022$\,m, range $[0.053,\, 0.180]$\,m)
because DA3 depth produces different corridor width estimates than
the sensor depth used by A1/A3.

\subsection{Class-Aware Inflation Integration (A5/A6)}
\label{sec:class-aware}

Added two new experiments using YOLO+SAM2 semantic segmentation for
per-class inflation radii:

\begin{table}[h]
\centering
\caption{Extended experiment matrix with class-aware inflation}
\label{tab:experiments-extended}
\begin{tabular}{llll}
\toprule
\textbf{Exp.} & \textbf{Obstacle Source} & \textbf{Inflation} & \textbf{Tests} \\
\midrule
A5 & Sensor + DA3 depth & Class-aware (YOLO+SAM2) & Per-class safety \\
A6 & DA3 depth only     & Class-aware (YOLO+SAM2) & Full vision-only \\
\bottomrule
\end{tabular}
\end{table}

\subsubsection{Implementation}

The class-aware strategy applies a separate inflation pass for each
semantic class present in the occupancy grid:

\begin{enumerate}[nosep]
  \item \textbf{Segmentation mask:} YOLO+SAM2 produces a 6-class pixel
    label map (H$\times$W, uint8, values 0--5) for each RGB frame.
  \item \textbf{Class grid:} \texttt{build\_class\_grid()} back-projects
    depth pixels to 3D, filters by height, and rasterises to the same
    bird's-eye grid as the occupancy, carrying each pixel's class label.
    When multiple classes map to the same cell, the most safety-critical
    class wins (person $>$ glass $>$ furniture $>$ other $>$ wall $>$ floor).
  \item \textbf{Per-class inflation:} \texttt{apply\_class\_aware\_inflation()}
    runs one distance-transform inflation pass per class, using the radii
    from Table~\ref{tab:class-radii}, then merges by taking the maximum
    cost at each cell.
\end{enumerate}

The effective inflation radius reported for A5/A6 is the weighted mean
of per-class radii across all lethal cells in each frame.

\subsubsection{Files Modified}

\begin{itemize}[nosep]
  \item \texttt{costmap\_builder.py}: added \texttt{back\_project\_depth\_with\_pixels()},
    \texttt{build\_class\_grid()}, and segmentation class constants.
  \item \texttt{inflation.py}: added \texttt{CLASS\_RADII} dict and
    \texttt{apply\_class\_aware\_inflation()} function with multi-pass
    per-class distance-transform inflation.
  \item \texttt{run\_costmap\_ablation.py}: added \texttt{--seg-dir} argument,
    A5/A6 experiments, segmentation mask loading with nearest-neighbour resize.
  \item \texttt{costmap\_ablation.slurm}: updated to run YOLO+SAM2 before
    ablation and pass \texttt{--seg-dir} when masks are available.
  \item \texttt{corridor\_sam2\_seg.slurm}: new standalone SLURM job for
    generating corridor segmentation masks on HPC.
\end{itemize}

\subsection{Run 3 Results (With Class-Aware Inflation)}

YOLO+SAM2 segmentation masks generated on HPC (SAM2-Hiera-Large +
YOLOv8x, 459 frames). All 459/459 masks loaded successfully.

\begin{table}[h]
\centering
\caption{Phase 1 corridor ablation --- Run 3 (459 frames, all 7 experiments)}
\label{tab:phase1-run3}
\begin{tabular}{lrrrrrrr}
\toprule
\textbf{Exp.} & \textbf{IoU} & \textbf{L-IoU} & \textbf{Det.Rate} & \textbf{FPR} & \textbf{Clear.} & \textbf{Radius} & \textbf{Time} \\
\midrule
Baseline & 1.000 & 1.000 & 1.000 & 0.000 & 0.249\,m & 0.090\,m & 13.8\,ms \\
A1       & 1.000 & 1.000 & 1.000 & 0.000 & 0.249\,m & 0.177\,m & 51.3\,ms \\
A2       & 0.379 & 0.295 & 1.000 & 0.053 & 0.171\,m & 0.090\,m & 71.7\,ms \\
A3       & 0.379 & 0.295 & 1.000 & 0.053 & 0.171\,m & 0.177\,m & 109.1\,ms \\
A4       & 0.279 & 0.226 & 0.767 & 0.053 & 0.175\,m & 0.165\,m & 108.5\,ms \\
A5       & 0.379 & 0.295 & 1.000 & 0.053 & 0.171\,m & 0.192\,m & 163.3\,ms \\
A6       & 0.279 & 0.226 & 0.767 & 0.053 & 0.175\,m & 0.197\,m & 149.6\,ms \\
\bottomrule
\end{tabular}
\end{table}

\textbf{Key findings (Run 3):}
\begin{enumerate}[nosep]
  \item \textbf{Class-aware inflation produces larger radii:} A5 mean radius
    $= 0.192$\,m (range 0.184--0.200\,m) vs A3 corridor\_width $= 0.177$\,m.
    A6 mean $= 0.197$\,m vs A4 $= 0.165$\,m. The per-class radii (person 0.30\,m,
    glass 0.20\,m, furniture 0.15\,m) pull the weighted average above the
    corridor-width estimate, providing semantically motivated safety margins.

  \item \textbf{A5 radius more stable than A4:} A5 std $= 0.0035$\,m vs
    A4 std $= 0.022$\,m. Class-aware inflation adapts to \emph{what} is detected
    rather than \emph{how narrow} the corridor appears, producing more consistent
    behaviour across frames.

  \item \textbf{IoU/detection unchanged between inflation strategies:} A5 matches
    A2/A3 (IoU 0.379, L-IoU 0.295, DetRate 1.000) and A6 matches A4 (IoU 0.279,
    L-IoU 0.226, DetRate 0.767). Inflation only changes the cost gradient
    around obstacles, not which cells are lethal.

  \item \textbf{Timing within budget:} worst case A5 at 163.3\,ms, still
    under the 200\,ms Nav2 local costmap update period. The multi-pass
    per-class inflation adds $\sim$54\,ms over A3.

  \item \textbf{Static corridor limits differentiation:} the corridor contains
    mostly walls and floor with few dynamic obstacles (no people during recording).
    The LILocBench \texttt{dynamics\_0} sequence (10 people moving) will better
    exercise the per-class radii, especially the person class (0.30\,m).
\end{enumerate}

\subsubsection{Class-Aware Inflation Radii Statistics}

\begin{table}[h]
\centering
\caption{Per-experiment inflation radius statistics (Run 3)}
\label{tab:radius-stats}
\begin{tabular}{lrrrr}
\toprule
\textbf{Exp.} & \textbf{Mean (m)} & \textbf{Std (m)} & \textbf{Min (m)} & \textbf{Max (m)} \\
\midrule
A1 & 0.1774 & 0.0047 & 0.1557 & 0.1800 \\
A3 & 0.1774 & 0.0047 & 0.1557 & 0.1800 \\
A4 & 0.1647 & 0.0215 & 0.0532 & 0.1800 \\
A5 & 0.1916 & 0.0035 & 0.1836 & 0.2000 \\
A6 & 0.1969 & 0.0025 & 0.1872 & 0.2016 \\
\bottomrule
\end{tabular}
\end{table}

% ============================================================================
\section{Phase 1: Offline LILocBench Ablation}
\label{sec:phase1-lilocbench}
% ============================================================================

\subsection{Dataset}

Selected sequences from LILocBench (Uni Bonn, 2025):

\begin{itemize}[nosep]
  \item \texttt{static\_0}: static office environment (13.9\,GB)
  \item \texttt{dynamics\_0}: 10 people moving ($\sim$3.8\,GB)
  \item \texttt{lt\_changes\_0}: rearranged furniture (10.0\,GB)
  \item \texttt{lt\_changes\_3}: hallway only, doors closed (10.1\,GB)
\end{itemize}

3$\times$ RealSense D455 cameras, 640$\times$480 @ 15\,Hz.
Subsampled to $\sim$3--5\,Hz.

\subsection{Results}

\textit{(To be filled after HPC run)}

% ============================================================================
\section{Phase 2: Live Corridor Ablation (Jetson)}
\label{sec:phase2}
% ============================================================================

\subsection{Setup}

DA3-Small via TensorRT on Jetson Orin Nano 8GB. ONNX/TRT build complete.
GerdsenAI ROS2 Wrapper publishes \texttt{/depth\_anything\_3/depth} at
25--40\,Hz.

Local development GPU: NVIDIA GTX 1060 6\,GB (PyTorch 2.7.1+cu118,
CUDA capability sm\_61).

\subsection{ROS2 Nodes Implemented}

\subsubsection{da3\_to\_pointcloud.py}

Converts DA3 monocular depth images to \texttt{PointCloud2} messages
for Nav2's \texttt{ObstacleLayer}:

\begin{itemize}[nosep]
  \item Subscribes to \texttt{/depth\_anything\_3/depth} (float32 Image)
    and \texttt{/camera/color/camera\_info} (for intrinsics)
  \item Back-projects valid depth pixels to 3D points in the camera
    optical frame
  \item Filters by height band (\texttt{min\_height}--\texttt{max\_height}
    above ground), matching the offline \texttt{costmap\_builder.py} logic
  \item Publishes \texttt{/da3/pointcloud} (\texttt{PointCloud2}) at
    camera frame rate
  \item Configurable downsample factor (default 4$\times$) for
    performance
\end{itemize}

\subsubsection{dynamic\_inflation.py}

Adjusts Nav2's local costmap \texttt{inflation\_radius} parameter
in real-time via \texttt{SetParameters} service:

\begin{itemize}[nosep]
  \item Subscribes to YOLO \texttt{Detection2DArray} and DA3 depth
  \item Strategies: \texttt{class\_aware} (person$=$0.30\,m,
    furniture$=$0.15\,m, other$=$0.12\,m), \texttt{corridor\_width},
    \texttt{min\_depth}, \texttt{fixed}
  \item Falls back to depth-based radius when no detections are present
  \item Updates at 5\,Hz, only when radius change exceeds 5\,mm
    (avoids parameter churn)
\end{itemize}

\subsection{Nav2 Configuration Variants}

\begin{itemize}[nosep]
  \item \texttt{nav2\_hardware.yaml} --- Baseline (scan only, fixed 0.09\,m)
  \item \texttt{nav2\_da3\_costmap.yaml} --- scan + DA3 \texttt{PointCloud2}
    fused obstacle layer (for A2/A3/A5)
  \item \texttt{nav2\_da3\_only.yaml} --- DA3 \texttt{PointCloud2} only,
    no LiDAR (for A4/A6)
\end{itemize}

Dynamic inflation (A1/A3/A5/A6) is handled by the
\texttt{dynamic\_inflation} node rather than separate YAML files,
keeping the configuration matrix manageable.

\subsection{Launch File}

\texttt{da3\_costmap.launch.py} brings up the perception stack:

\begin{lstlisting}
# Scan + DA3, fixed inflation (A2):
ros2 launch rc_hardware_bringup da3_costmap.launch.py

# Scan + DA3, class-aware inflation (A5):
ros2 launch rc_hardware_bringup da3_costmap.launch.py \
    inflation_strategy:=class_aware

# DA3-only, class-aware (A6):
ros2 launch rc_hardware_bringup da3_costmap.launch.py \
    inflation_strategy:=class_aware
# (with nav2_da3_only.yaml as Nav2 params)
\end{lstlisting}

\subsection{Results}

\textit{(To be filled)}

% ============================================================================
\section{Pipeline and Files}
\label{sec:files}
% ============================================================================

\begin{longtable}{lp{9cm}}
\toprule
\textbf{File} & \textbf{Description} \\
\midrule
\endfirsthead
\toprule
\textbf{File} & \textbf{Description} \\
\midrule
\endhead
\texttt{ml\_pipeline/costmap\_builder.py} & Depth image to 2D occupancy grid.
  Back-projects depth to 3D, filters by height band, rasterises to bird's-eye
  grid. Also builds class grids from YOLO+SAM2 segmentation masks. \\
\texttt{ml\_pipeline/inflation.py} & Fixed inflation (Nav2 replication) plus
  dynamic strategies: corridor\_width, min\_depth, and class\_aware
  (per-class inflation with YOLO+SAM2 semantics). \\
\texttt{ml\_pipeline/run\_costmap\_ablation.py} & Orchestrates all 7 experiments
  (Baseline + A1--A6). Loads sensor + DA3 depth + segmentation, builds grids,
  applies inflation, computes metrics, outputs CSV + JSON summaries. \\
\texttt{ml\_pipeline/costmap\_ablation.slurm} & SLURM job for full HPC ablation
  including YOLO+SAM2 segmentation (corridor + LILocBench). \\
\texttt{ml\_pipeline/corridor\_sam2\_seg.slurm} & Standalone SLURM job for
  running YOLO+SAM2 on corridor frames only. \\
\texttt{ml\_pipeline/extract\_lilocbench.py} & Extracts LILocBench frames to
  manifest format (RGB + depth .npy + manifest.jsonl). \\
\texttt{ml\_pipeline/step1\_download\_lilocbench.sh} & Downloads LILocBench
  sequences to HPC scratch. \\
\midrule
\multicolumn{2}{l}{\textbf{Phase 2: ROS2 Live Integration (NCHSB)}} \\
\midrule
\texttt{rc\_hardware\_bringup/da3\_to\_pointcloud.py} & ROS2 node:
  DA3 depth image $\to$ \texttt{PointCloud2} for Nav2 \texttt{ObstacleLayer}.
  Back-projects, height-filters, publishes at camera rate. \\
\texttt{rc\_hardware\_bringup/dynamic\_inflation.py} & ROS2 node:
  adjusts Nav2 \texttt{inflation\_radius} via \texttt{SetParameters}
  based on YOLO detections (class\_aware, corridor\_width, min\_depth). \\
\texttt{config/nav2\_da3\_costmap.yaml} & Nav2 config with scan + DA3
  \texttt{PointCloud2} fused obstacle layer. \\
\texttt{config/nav2\_da3\_only.yaml} & Nav2 config with DA3-only
  obstacle layer (no LiDAR). \\
\texttt{launch/da3\_costmap.launch.py} & Launch file for DA3 perception
  stack with configurable inflation strategy. \\
\bottomrule
\end{longtable}

% ============================================================================
\section{Timeline}
\label{sec:timeline}
% ============================================================================

\begin{longtable}{lp{10cm}}
\toprule
\textbf{Date} & \textbf{Milestone} \\
\midrule
\endfirsthead
Mar 1, 2026 & Branch \texttt{da3-costmap-ablation} created off \texttt{hardware-integration} \\
Mar 1, 2026 & Costmap builder, inflation module, and ablation runner implemented \\
Mar 1, 2026 & SLURM job script created for HPC execution \\
Mar 1, 2026 & Run 1 (5 experiments, corridor 459 frames), identified dynamic radius issues \\
Mar 1, 2026 & Fix 1--3: inverted radius scaling, cost-threshold metrics, class-aware radii \\
Mar 1, 2026 & Run 2 (post-fix), confirmed dynamic inflation now adaptive \\
Mar 2, 2026 & Class-aware inflation (A5/A6): \texttt{build\_class\_grid()},
              \texttt{apply\_class\_aware\_inflation()} multi-pass per-class inflation \\
Mar 2, 2026 & \texttt{corridor\_sam2\_seg.slurm}: HPC job for YOLO+SAM2 on corridor frames \\
Mar 2, 2026 & YOLO+SAM2 run on HPC (459 masks), Run 3 with all 7 experiments complete \\
Mar 2, 2026 & PyTorch 2.7.1+cu118 installed locally (GTX 1060 sm\_61 support) \\
Mar 2, 2026 & Phase 2 ROS2 nodes: \texttt{da3\_to\_pointcloud.py},
              \texttt{dynamic\_inflation.py} \\
Mar 2, 2026 & Nav2 configs: \texttt{nav2\_da3\_costmap.yaml} (fused),
              \texttt{nav2\_da3\_only.yaml} (DA3-only) \\
Mar 2, 2026 & \texttt{da3\_costmap.launch.py}: perception launch with
              configurable inflation strategy \\
            & \textit{(More entries to be added as work progresses)} \\
\bottomrule
\end{longtable}

\end{document}
